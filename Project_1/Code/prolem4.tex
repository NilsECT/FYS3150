\documentclass[english,notitlepage]{article}  % defines the basic parameters of the document
\usepackage[T1]{fontenc} %for å bruke æøå
\usepackage[utf8]{inputenc}
\usepackage{graphicx} %for å inkludere grafikk
\usepackage{mathpazo}
\usepackage[norsk]{babel}
% Standard stuff
\usepackage{amsmath,graphicx,varioref,verbatim,amsfonts,geometry,gensymb, multirow}
% colors in text
\usepackage[usenames,dvipsnames,svgnames,table]{xcolor}
% Hyper refs
\usepackage[colorlinks=true,allcolors=black]{hyperref}

\usepackage{caption}
\usepackage{enumitem}
\usepackage{tikz}             % draw figures manually
\usepackage{subfigure}        % imports a lot of cool and useful figure commands
\usepackage{float}
\usepackage{circuitikz}

\usepackage{csquotes}
\usepackage[backend=biber,style=alphabetic,sorting=ynt]{biblatex} %Imports biblatex package
\addbibresource{referanser.bib}

\begin{document}
\title{Project 1 - Problem 4}
\date{\today}

\maketitle

We have the following discretized expression:

\begin{equation}\label{eq:second_derr}
    \left[\frac{u_{i+1} - 2u_i + u_{i-1}}{h^2} + O(h^2) = f_i\right]
\end{equation}

and we define $v_i = \frac{u_{i+1} - 2u_i + u_{i-1}}{h^2} \approx f_i$ and $\vec{v}$ to be the vector containing all $v_i$.

With this in mind we can set up a matrix equation with a tri-diagonal matrix, $\boldsymbol{A}\vec{v} = \vec{g}$, as
\begin{equation}\label{eq:mat_Avg}
    \begin{bmatrix}
        2 & -1 & 0 & 0 \\
        -1 & 2 & -1 & 0 \\
        0 & -1 & 2 & -1 \\
        0 & 0 & -1 & 2
    \end{bmatrix} \begin{bmatrix}
        v_1\\
        v_2\\
        v_3\\
        v_4
    \end{bmatrix} = \begin{bmatrix}
        g_1\\
        g_2\\
        g_3\\
        g_4
    \end{bmatrix}
\end{equation}

We can now by compute $\boldsymbol{A}\vec{v}$ and compare the each row to \hyperref[eq:second_derr]{equation \ref*{eq:second_derr}}. We can multiply \hyperref[eq:second_derr]{equation \ref*{eq:second_derr}} by $h^2$ on both sides and define the right-hand side to be $g_i$ defining $\vec{g}$ appropriately we get the following:

\begin{equation}\label{eq:mat_Vhf}
    \begin{bmatrix}
        2v_1 & -v_2 & 0 & 0 \\
        -v_1 & 2v_2 & -v_3 & 0 \\
        0 & -v_2 & 2v_3 & -v_4 \\
        0 & 0 & -v_3 & 2v_4
    \end{bmatrix} = \begin{bmatrix}
        g_1 \equiv h^2 f_1\\
        g_2 \equiv h^2 f_2\\
        g_3 \equiv h^2 f_3\\
        g_4 \equiv h^2 f_4
    \end{bmatrix}
\end{equation}

So we can represent the discretized equation of the second derrivative as a matrix equation.


\end{document}
