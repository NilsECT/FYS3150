\documentclass[english,notitlepage]{revtex4-1}  % defines the basic parameters
%of the document For preview: skriv i terminal: latexmk -pdf -pvc filnavn



% if you want a single-column, remove reprint

% allows special characters (including æøå)
\usepackage[utf8]{inputenc}
%\usepackage[english]{babel}

%% note that you may need to download some of these packages manually, it
%depends on your setup. % I recommend downloading TeXMaker, because it includes
%a large library of the most common packages.

\usepackage{physics,amssymb}  % mathematical symbols (physics imports amsmath)
\usepackage{amsmath}
\usepackage{graphicx}         % include graphics such as plots
\usepackage{xcolor}           % set colors
\usepackage{hyperref}         % automagic cross-referencing (this is GODLIKE)
\usepackage{listings}         % display code
\usepackage{subfigure}        % imports a lot of cool and useful figure commands
\usepackage{float}
%\usepackage[section]{placeins}
\usepackage{algorithm}
\usepackage[noend]{algpseudocode}
\usepackage{subfigure}
\usepackage{tikz}
\usetikzlibrary{quantikz}
% defines the color of hyperref objects Blending two colors:  blue!80!black  =
% 80% blue and 20% black
\hypersetup{ % this is just my personal choice, feel free to change things
    colorlinks, linkcolor={red!50!black}, citecolor={blue!50!black},
    urlcolor={blue!80!black}}


\begin{document}

\title{Problem 9}      % self-explanatory
\author{Frida}          % self-explanatory
\date{\today}                             % self-explanatory
\noaffiliation                            % ignore this, but keep it.

\maketitle

\section{Problem 9}

\subsection{Specialize the algorithm}


We want to specialize the algorithm from Problem 6 for the special case where $\boldsymbol{A}$ is specified by the signature $(-1, 2, -1)$. This menas that $\vec{a}$ and $\vec{c}$ are vectors of length $n-1$, consisting only of the value $-1$, and $\vec{b}$ is an $n$-length vector filled with the value $2$.

Substituting every $a_i, c_i$ with $-1$ and $b_i$ with $2$, we obtain the following equations for

\begin{equation}
  \begin{split}
    \tilde{b}_1 &= b_1 = 2 \\
    \tilde{b}_i &= b_i - \frac{a_i}{\tilde{b}_{i-1}} c_{i-1} = 2 - \frac{1}{\tilde{b}_{i-1}} \\ \label{eqn:specalg_b}
   \end{split}
\end{equation}

\begin{equation}
  \begin{split}
    \tilde{g}_1 &= g_1 \\
    \tilde{g}_i &= g_i - \frac{a_i}{\tilde{b}_{i-1}} \tilde{g}_{i-1} = g_i + \frac{\tilde{g}_{i-1}}{\tilde{b}_{i-1}} \\ \label{eqn:specalg_g}
   \end{split}
\end{equation}

Similarly, for the backward substitution we get the following equations:

\begin{equation}
  \begin{split}
    v_n &= \frac{\tilde{g}_n}{\tilde{b_n}} \\
    v_i &= \frac{\tilde{g}_i - c_i v_{i+1}}{\tilde{b}_i} = \frac{\tilde{g}_i + v_{i+1}}{\tilde{b}_i} \\ \label{eqn:specalg_v}
   \end{split}
\end{equation}

\subsection{Count the number of FLOPs in specialized algorithm}

We count the number of FLOPs required for the special algorithm.

When calculating every element of $\vec{\tilde{b}}$, we see from \eqref{eqn:specalg_b} that we perform 2 FLOPs (consisting of addition and division) for every $i \in [1, N-1]$. The initial element $b_1$ requires no FLOPs. Exactly the same holds for calculating $\vec{\tilde{g}}$ in \eqref{eqn:specalg_g}.

However, for the backward substitution in \eqref{eqn:specalg_v}, we see that in addition to the $2(n-1)$ FLOPs, we need one more for the last term $v_n$.

In total, we require

\begin{equation}
  3 ~ \cdotp \big[ 2(n-1)\big] + 1 = 6n-5 \text{ FLOPs}
\end{equation}

\subsection{Implement algorithm in code}

Lastly, we write code that implements the algorithm. The code lies in \lstinline{Problem9.cpp}.


\end{document}
