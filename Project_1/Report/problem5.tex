\documentclass[english,notitlepage]{revtex4-1}  % defines the basic parameters of the document
%For preview: skriv i terminal: latexmk -pdf -pvc filnavn



% if you want a single-column, remove reprint

% allows special characters (including æøå)
\usepackage[utf8]{inputenc}
%\usepackage[english]{babel}

%% note that you may need to download some of these packages manually, it depends on your setup.
%% I recommend downloading TeXMaker, because it includes a large library of the most common packages.

\usepackage{physics,amssymb}  % mathematical symbols (physics imports amsmath)
\include{amsmath}
\usepackage{graphicx}         % include graphics such as plots
\usepackage{xcolor}           % set colors
\usepackage{hyperref}         % automagic cross-referencing (this is GODLIKE)
\usepackage{listings}         % display code
\usepackage{subfigure}        % imports a lot of cool and useful figure commands
\usepackage{float}
%\usepackage[section]{placeins}
\usepackage{algorithm}
\usepackage[noend]{algpseudocode}
\usepackage{subfigure}
\usepackage{tikz}
\usetikzlibrary{quantikz}
% defines the color of hyperref objects
% Blending two colors:  blue!80!black  =  80% blue and 20% black
\hypersetup{ % this is just my personal choice, feel free to change things
    colorlinks,
    linkcolor={red!50!black},
    citecolor={blue!50!black},
    urlcolor={blue!80!black}}

%% Defines the style of the programming listing
%% This is actually my personal template, go ahead and change stuff if you want



%% USEFUL LINKS:
%%
%%   UiO LaTeX guides:        https://www.mn.uio.no/ifi/tjenester/it/hjelp/latex/
%%   mathematics:             https://en.wikibooks.org/wiki/LaTeX/Mathematics

%%   PHYSICS !                https://mirror.hmc.edu/ctan/macros/latex/contrib/physics/physics.pdf

%%   the basics of Tikz:       https://en.wikibooks.org/wiki/LaTeX/PGF/Tikz
%%   all the colors!:          https://en.wikibooks.org/wiki/LaTeX/Colors
%%   how to draw tables:       https://en.wikibooks.org/wiki/LaTeX/Tables
%%   code listing styles:      https://en.wikibooks.org/wiki/LaTeX/Source_Code_Listings
%%   \includegraphics          https://en.wikibooks.org/wiki/LaTeX/Importing_Graphics
%%   learn more about figures  https://en.wikibooks.org/wiki/LaTeX/Floats,_Figures_and_Captions
%%   automagic bibliography:   https://en.wikibooks.org/wiki/LaTeX/Bibliography_Management  (this one is kinda difficult the first time)
%%   REVTeX Guide:             http://www.physics.csbsju.edu/370/papers/Journal_Style_Manuals/auguide4-1.pdf
%%
%%   (this document is of class "revtex4-1", the REVTeX Guide explains how the class works)


%% CREATING THE .pdf FILE USING LINUX IN THE TERMINAL
%%
%% [terminal]$ pdflatex template.tex
%%
%% Run the command twice, always.
%% If you want to use \footnote, you need to run these commands (IN THIS SPECIFIC ORDER)
%%
%% [terminal]$ pdflatex template.tex
%% [terminal]$ bibtex template
%% [terminal]$ pdflatex template.tex
%% [terminal]$ pdflatex template.tex
%%
%% Don't ask me why, I don't know.

\begin{document}

\title{Problem 5}      % self-explanatory
\author{Frida}          % self-explanatory
\date{\today}                             % self-explanatory
\noaffiliation                            % ignore this, but keep it.


\maketitle

\textit{List a link to your github repository here!}

\section*{Problem 5}

  We let $\vec{v}^* = [v^*_1, v^*_2, ... v^*_m]$ denote the vector of length $m$ that represents the complete solution of the discretized Poisson equation. The corresponding $x$ values are contained in $\vec{x} = [x_1, x_2, ..., x_m]$, with length $m$. We let \textbf{A} be an $n \times n$ matrix.

  \subsection*{a)}

    Writing out our matrix equation $\boldsymbol{A} \vec{v} = \vec{g}$, we get

    \begin{equation}
      \begin{bmatrix}
          2 & -1 & 0 & ... & 0 \\
          -1 & 2 & ... & ... & ... \\
          0 & ... & ... & -1 & 0 \\
          ... & ... & -1 & 2 & -1 \\
          0 & ... & 0 & -1 & 2 \\
      \end{bmatrix} \begin{bmatrix}
          v_1 \\
          v_2 \\
          ... \\
          v_{n-1} \\
          v_n \\
      \end{bmatrix} =
      \begin{bmatrix}
          g_1 \\
          g_2 \\
          ... \\
          g_{n-1} \\
          g_n \\
      \end{bmatrix}
    \end{equation}

    Further, by writing out the multiplication for the first element, we get

    \begin{equation}
      2 v_1 - v_2 = g_1  \label{eqn:v_0}
    \end{equation}

    But we see, from the discretized version of the Poisson equation, \textbf{REFERER TIL den ultednigngng } that this corresponds to the \emph{second} term in our solution $\vec{v}^*$, meaning the first element \emph{after} the boundary term (which in our case is $u(0) = v^*_1 = 0$). Including the boundary term, we get $2 v_1 - v_2 = -v_{0} + 2 v_1 - v_2 = g_1 $.

    Further, we write out the last element of the matrix equation:

    \begin{equation}
      -v_{n-1} + 2 v_n = g_n \label{eqn:v_n}
    \end{equation}

    Once again, we see that $v_n$ corresponds to the second-to-last element of $v^*$ (meaning the $v^*_{m-1})$, and the equation holds because the $m$th entry of $\vec{v}^*$ is $v^*_m = 0 = u(1)$.

    %If we denote the boundary term by $v_0^*$, then we have the correspondence $v_0 \rightleftarrow v^*_1$. Similarly, $g_0 \rightleftarrow f_1$.

    In general, we find there is a correspondence $v_i \longleftrightarrow v^*_{i+1}$ for $i \in \{1, ..., n\}$.

    As $\vec{v}$ has length $n$ and $\vec{v}^*$ has length $m$, we must have that $m = n+2$, where the additional two terms are due to the boundary terms.


  \subsection*{b)}

    As $m = n+2$, we see that $\vec{v}$ only contains the 'middle' part of $\vec{v}^*$, meaning all elements excluding the two boundary terms.

    For the right hand side, we have to remember that $g_i$ corresponds to $f(x_{i+1})$.


\end{document}
