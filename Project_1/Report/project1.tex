\documentclass[english,notitlepage]{article}  % defines the basic parameters of the document
\usepackage[T1]{fontenc} %for å bruke æøå
\usepackage[utf8]{inputenc}
\usepackage{graphicx} %for å inkludere grafikk
\usepackage{mathpazo}
\usepackage[norsk]{babel}
% Standard stuff
\usepackage{amsmath,graphicx,varioref,verbatim,amsfonts,geometry,gensymb, multirow}
% colors in text
\usepackage[usenames,dvipsnames,svgnames,table]{xcolor}
% Hyper refs
\usepackage[colorlinks=true,allcolors=black]{hyperref}

\usepackage{caption}
\usepackage{enumitem}
\usepackage{tikz}             % draw figures manually
\usepackage{subfigure}        % imports a lot of cool and useful figure commands
\usepackage{float}
\usepackage{circuitikz}
\usepackage{listings}

\usepackage{csquotes}
\usepackage[backend=biber,style=alphabetic,sorting=ynt]{biblatex} %Imports biblatex package
\addbibresource{referanser.bib}

\begin{document}

\title{FYS3150\\Project 1}
\author{Brage Andreas Trefjord,\\Sigurd Sønvisen Vargdal,\\Nils Enric Canut Taugbøl,\\Frida Oleivsgard Sørensen}
\maketitle

\textit{GitHub repository:} \texttt{\url{https://github.com/NilsECT/FYS3150}}

\section*{Problem 1}

  The one-dimensional Poisson equation can be written as

  \begin{equation}
    -\frac{d^2 u}{dx^2} = f(x) \label{eqn:poisson}
  \end{equation}

  where $f(x)$, the source term, is known. We assume a setup such that the source term is $f(x) = 100e^{-10x}$, $x \in [0, 1]$, and the boundary conditions are $u(0) = 0$ and $u(1)=0$.


  We want to check analytically that an exact solution to \hyperref[eqn:poisson]{equation \ref*{eqn:poisson}} can be given by

  \begin{equation}
    u(x) = 1 - (1-e^{-10})x - e^{-10x} \label{eqn:u(x)}
  \end{equation}

  We differentiate $u(x)$ twice and find that

  \begin{equation}
    \begin{split}
      \frac{d^2}{dx^2} u(x) &= \bigg( 1 - (1-e^{-10} ) x - e^{-10x} \bigg) \\
      &=  \frac{d}{dx} \bigg( 1-e^{-10} + 10 e^{-10x} \bigg) \\
      &=  - 100 e^{-10x} \\
      &= -f(x) \\
    \end{split}
  \end{equation}

  as we wanted. In addition, we check whether the boundary conditions are fulfilled:

  \begin{equation}
    \begin{split}
      u(0) &= \big( 1 - e^0 \big) = 1-1 \\
      &= 0 \\
      u(1) &= 1-1+e^{-10} - e^{-10} \\
      &= 0 \\
    \end{split}
  \end{equation}

  This means $u(x)$ in \hyperref[eqn:u(x)]{equation \ref*{eqn:u(x)}}
   is a solution to our specific setup.




\section*{Problem 3: Deriving the discretized Poisson equation.}

    Let's start by deriving the discretized version of $\frac{d^2u}{dx^2}$. First,
    we taylor expand the function $u(x+h)$ around the point $x$. We get:

    \begin{align*}
        u(x+h) &= \sum_{n=0}^{\infty} \frac{u^{(n)}(x)}{n!} h^n
        \\
        u(x+h) &= u(x) + u'(x) h + \frac{1}{2} u''(x) h^2 + \frac{1}{6} u'''(x) h^3 + O_1(h^4)
    \end{align*}

    where $O_1(h^2)$ is the remainder of the expansion, and $O(h) \equiv
    \frac{O_1(h^2)}{h}$. Now let's also Taylor expand $u(x-h)$ (again, around the
    point $x$).

    \begin{align*}
        u(x-h) &= \sum_{n=0}^{\infty} \frac{u^{(n)}(x)}{n!} (-h)^n
        \\
        u(x-h) &= u(x) - u'(x) h + \frac{1}{2} u''(x) h^2 - \frac{1}{6} u'''(x) h^3 + O_2(h^4)
    \end{align*}

    If we now add these two expansions together, we get:

    \begin{align*}
        u(x+h) + u(x-h) &= u(x) + u'(x) h + \frac{1}{2} u''(x) h^2 + \frac{1}{6} u'''(x) h^3
        + O_1(h^4)
        \\ & \hspace*{30pt} + u(x) - u'(x) h + \frac{1}{2} u''(x) h^2 - \frac{1}{6} u'''(x) h^3 + O_2(h^4)
        \\
        u(x+h) + u(x-h) &= 2u(x) + u''(x) h^2 + O_1(h^4) + O_2(h^4)
        \\
        u''(x) &= \frac{u(x+h) - 2u(x) + u(x-h)}{h^2} - \frac{O_1(h^4) + O_2(h^4)}{h^2}
        \\
        u''(x) &= \frac{u(x+h) - 2u(x) + u(x-h)}{h^2} + O(h^2)
    \end{align*}

    This can be discretized by letting $v_i \approx u(x)$, $v_{i+1} \approx u(x+h)$
    and $v_{i-1} \approx u(x-h)$. Since this is an approximation we can ignore the
    remainder, and our result becomes

    \begin{equation*}
        v''_i = \frac{v_{i+1} - 2v_i + v_{i-1}}{h^2}
    \end{equation*}

    Here $x$ is discretized as $i h$, where $h$ is the step length, and $i$ is the
    number of steps to reach the $x$ value. Using this, the discretization of the
    forcing term $f(x) = 100 e^{-10 x}$ becomes $f_i = 100 e^{-10 ih}$. Using this
    together with the discretized version of $v''(x)$ we get the complete
    discretized Poisson equation:

    \begin{equation}
        -\frac{v_{i+1} - 2v_i + v_{i-1}}{h^2} = 100 e^{-10 i h}
    \end{equation}

\section*{Problem 4}

    We have the following discretized expression:

    \begin{equation}\label{eq:second_derr}
        \left[\frac{u_{i+1} - 2u_i + u_{i-1}}{h^2} + O(h^2) = f_i\right]
    \end{equation}

    and we define $v_i = \frac{u_{i+1} - 2u_i + u_{i-1}}{h^2} \approx f_i$ and $\vec{v}$ to be the vector containing all $v_i$.

    With this in mind we can set up a matrix equation with a tri-diagonal matrix, $\boldsymbol{A}\vec{v} = \vec{g}$, as
    \begin{equation}\label{eq:mat_Avg}
        \begin{bmatrix}
            2 & -1 & 0 & 0 \\
            -1 & 2 & -1 & 0 \\
            0 & -1 & 2 & -1 \\
            0 & 0 & -1 & 2
        \end{bmatrix} \begin{bmatrix}
            v_1\\
            v_2\\
            v_3\\
            v_4
        \end{bmatrix} = \begin{bmatrix}
            g_1\\
            g_2\\
            g_3\\
            g_4
        \end{bmatrix}
    \end{equation}

    We can now by compute $\boldsymbol{A}\vec{v}$ and compare the each row to \hyperref[eq:second_derr]{equation \ref*{eq:second_derr}}. We can multiply \hyperref[eq:second_derr]{equation \ref*{eq:second_derr}} by $h^2$ on both sides and define the right-hand side to be $g_i$ defining $\vec{g}$ appropriately we get the following:

    \begin{equation}\label{eq:mat_Vhf}
        \begin{bmatrix}
            2v_1 & -v_2 & 0 & 0 \\
            -v_1 & 2v_2 & -v_3 & 0 \\
            0 & -v_2 & 2v_3 & -v_4 \\
            0 & 0 & -v_3 & 2v_4
        \end{bmatrix} = \begin{bmatrix}
            g_1 \equiv h^2 f_1\\
            g_2 \equiv h^2 f_2\\
            g_3 \equiv h^2 f_3\\
            g_4 \equiv h^2 f_4
        \end{bmatrix}
    \end{equation}

    So we can represent the discretized equation of the second derrivative as a matrix equation.

\section*{Problem 5}

  We let $\vec{v}^* = [v^*_1, v^*_2, ... v^*_m]$ denote the vector of length $m$ that represents the complete solution of the discretized Poisson equation. The corresponding $x$ values are contained in $\vec{x} = [x_1, x_2, ..., x_m]$, with length $m$. We let \textbf{A} be an $n \times n$ matrix.

  \subsection*{Finding the relation between lengths $m$ and $n$}

    Writing out our matrix equation $\boldsymbol{A} \vec{v} = \vec{g}$, we get

    \begin{equation}
      \begin{bmatrix}
          2 & -1 & 0 & ... & 0 \\
          -1 & 2 & ... & ... & ... \\
          0 & ... & ... & -1 & 0 \\
          ... & ... & -1 & 2 & -1 \\
          0 & ... & 0 & -1 & 2 \\
      \end{bmatrix} \begin{bmatrix}
          v_1 \\
          v_2 \\
          ... \\
          v_{n-1} \\
          v_n \\
      \end{bmatrix} =
      \begin{bmatrix}
          g_1 \\
          g_2 \\
          ... \\
          g_{n-1} \\
          g_n \\
      \end{bmatrix}
    \end{equation}

    Further, by writing out the multiplication for the first element, we get

    \begin{equation}
      2 v_1 - v_2 = g_1  \label{eqn:v_0}
    \end{equation}

    But we see, from the discretized version of the Poisson equation in \hyperref[eq:mat_Vhf]{equation \ref*{eq:mat_Vhf}} that $v_1$ corresponds to the \emph{second} term in our solution $\vec{v}^*$, meaning the first element \emph{after} the boundary term (which in our case is $u(0) = v^*_1 = 0$). Including the boundary term, we get $2 v_1 - v_2 = -v_{0} + 2 v_1 - v_2 = g_1 $.

    Further, we write out the last element of the matrix equation:

    \begin{equation}
      -v_{n-1} + 2 v_n = g_n \label{eqn:v_n}
    \end{equation}

    Once again, we see that $v_n$ corresponds to the second-to-last element of $v^*$ (meaning the $v^*_{m-1})$, and the equation holds because the $m$th entry of $\vec{v}^*$ is $v^*_m = 0 = u(1)$.

    %If we denote the boundary term by $v_0^*$, then we have the correspondence $v_0 \rightleftarrow v^*_1$. Similarly, $g_0 \rightleftarrow f_1$.

    In general, we find there is a correspondence $v_i \longleftrightarrow v^*_{i+1}$ for $i \in \{1, ..., n\}$.

    As $\vec{v}$ has length $n$ and $\vec{v}^*$ has length $m$, we must have that $m = n+2$, where the additional two terms are due to the boundary terms.


  \subsection*{Which part of the complete solution do we get?}

    As $m = n+2$, we see that $\vec{v}$ only contains the 'middle' part of $\vec{v}^*$, meaning all elements excluding the two boundary terms.

    For the right hand side, we have to remember that $g_i$ corresponds to $f(x_{i+1})$.

\section*{Problem 6}

    The general expression for $\boldsymbol{A}\vec{v} = \vec{g}$ with $\vec{a}$, $\vec{b}$ and $\vec{c}$ being the sub-, main- and supdiagonal respectively is:
    \begin{equation}\label{eq:mat_Avg}
        \begin{bmatrix}
            b_1 & c_2 & 0 & 0 \\
            a_1 & b_2 & c_3 & 0 \\
            0 & a_2 & b_3 & c_4 \\
            0 & 0 & a_3 & b_4
        \end{bmatrix} \begin{bmatrix}
            v_1\\
            v_2\\
            v_3\\
            v_4
        \end{bmatrix} = \begin{bmatrix}
            g_1\\
            g_2\\
            g_3\\
            g_4
        \end{bmatrix}
    \end{equation}

    We can solve this with Gaussian elimination in two general steps: \hyperref[sec:forward]{[1]} Forward substitution and \hyperref[sec:backward]{[2]} Backward substitution.

    \subsection*{Forward substitution}\label{sec:forward}

    The goal here is to get an upper triangular matrix, that is we want to do the follwoing:
    \begin{equation*}
        \begin{bmatrix}
            \cdot & \cdot & \cdot & \cdot \\
            \cdot & \cdot & \cdot & \cdot \\
            \cdot & \cdot & \cdot & \cdot \\
            \cdot & \cdot & \cdot & \cdot
        \end{bmatrix} \begin{bmatrix}
            \cdot\\
            \cdot\\
            \cdot\\
            \cdot
        \end{bmatrix} = \begin{bmatrix}
            \cdot\\
            \cdot\\
            \cdot\\
            \cdot
        \end{bmatrix}
        \rightarrow
        \begin{bmatrix}
            \cdot & \cdot & \cdot & \cdot \\
            0 & \cdot & \cdot & \cdot \\
            0 & 0 & \cdot & \cdot \\
            0 & 0 & 0 & \cdot
        \end{bmatrix} \begin{bmatrix}
            \cdot\\
            \cdot\\
            \cdot\\
            \cdot
        \end{bmatrix} = \begin{bmatrix}
            \cdot\\
            \cdot\\
            \cdot\\
            \cdot
        \end{bmatrix}
    \end{equation*}

    So we want to get rid of all $a_i$ entries in \hyperref[eq:mat_Avg]{the matrix equation (\ref*{eq:mat_Avg})}. We will number and denote the rows with roman numerals.

    \begin{gather}
        \RN{2} = \RN{2} - \frac{a_2}{b_1}\RN{1}\\
        \text{The diagonal new entries will be denoted } \tilde{b_i} \nonumber \\
        \RN{3} = \RN{3} - \frac{a_3}{\tilde{b_2}}\RN{2}\\
        \RN{4} = \RN{4} - \frac{a_4}{\tilde{b_3}}\RN{3}\\
    \end{gather}

    This gives us the following algorithm for precdure
    \begin{gather}\label{eq:for}
        \tilde{b_1} = b_1 \\
        \tilde{b_i} = b_i - \frac{a_i}{\tilde{b_{i-1}}}c_{i-1}\\\label{line:for_b}
        \tilde{g_1} = g_1 \\
        \tilde{g_i} = g_i - \frac{a_i}{\tilde{b_{i-1}}}\tilde{g_{i-1}}\\\label{line:for_g}
    \end{gather}
    This will be for $i = 2, 3, ..., n$ for a general tridiagonal matrix.

  \subsection*{Backward substitution}\label{sec:backward}

    Now that we have done the forward substitution we want to find the solution to all $v_i$ from the following equation:

    \begin{equation}
        \begin{bmatrix}
            \tilde{b_1} & c_2 & 0 & 0 \\
            0 & \tilde{b_2} & c_3 & 0 \\
            0 & 0 & \tilde{b_3} & c_4 \\
            0 & 0 & 0 & \tilde{b_4}
        \end{bmatrix} \begin{bmatrix}
            v_1\\
            v_2\\
            v_3\\
            v_4
        \end{bmatrix} = \begin{bmatrix}
            \tilde{g_1}\\
            \tilde{g_2}\\
            \tilde{g_3}\\
            \tilde{g_4}
        \end{bmatrix}
    \end{equation}

    we quickly see that $v_4 = \frac{\tilde{g_4}}{\tilde{b_4}}$ and to get the remaining $v_i$ we move up with the following algorithm (we include the case of $v_n$ to generalize the algorithm):
    \begin{gather}\label{eq:back}
        v_n = \frac{\tilde{g_n}}{\tilde{b_n}}\\
        v_i = \frac{\tilde{g_i} - c_iv_{i-1}}{\tilde{b_i}}\\
    \end{gather}
    for $i = n-1, n-2, ..., 2, 1$.

  \subsection*{FLOPs}

    We can see that \hyperref[eq:for]{algorithm \ref*{eq:for}} requires $2(n-1)$ floating point operations. $n-1$ FLOPs for \hyperref[line:for_b]{equation (\ref*{line:for_b})} and \hyperref[line:for_g]{equation (\ref*{line:for_g})} each. From \hyperref[eq:back]{equation \ref*{eq:back}} we can count $3(n-1) + 1$ FLOPs. This gives us a total of $5n - 4$ FLOPs to solve $\boldsymbol{A}\vec{v} = \vec{g}$ with a tridiagonal matrix of dimentions $n\times n$.

\section*{Problem 9}

  \subsection*{Specialize the algorithm}


    We want to specialize the algorithm from Problem 6 for the special case where $\boldsymbol{A}$ is specified by the signature $(-1, 2, -1)$. This menas that $\vec{a}$ and $\vec{c}$ are vectors of length $n-1$, consisting only of the value $-1$, and $\vec{b}$ is an $n$-length vector filled with the value $2$.

    Substituting every $a_i, c_i$ with $-1$ and $b_i$ with $2$, we obtain the following equations for

    \begin{equation}
      \begin{split}
        \tilde{b}_1 &= b_1 = 2 \\
        \tilde{b}_i &= b_i - \frac{a_i}{\tilde{b}_{i-1}} c_{i-1} = 2 - \frac{1}{\tilde{b}_{i-1}} \\ \label{eqn:specalg_b}
       \end{split}
    \end{equation}

    \begin{equation}
      \begin{split}
        \tilde{g}_1 &= g_1 \\
        \tilde{g}_i &= g_i - \frac{a_i}{\tilde{b}_{i-1}} \tilde{g}_{i-1} = g_i + \frac{\tilde{g}_{i-1}}{\tilde{b}_{i-1}} \\ \label{eqn:specalg_g}
       \end{split}
    \end{equation}

    Similarly, for the backward substitution we get the following equations:

    \begin{equation}
      \begin{split}
        v_n &= \frac{\tilde{g}_n}{\tilde{b_n}} \\
        v_i &= \frac{\tilde{g}_i - c_i v_{i+1}}{\tilde{b}_i} = \frac{\tilde{g}_i + v_{i+1}}{\tilde{b}_i} \\ \label{eqn:specalg_v}
       \end{split}
    \end{equation}

  \subsection*{Number of FLOPs in specialized algorithm}

    We count the number of FLOPs required for the special algorithm.

    When calculating every element of $\vec{\tilde{b}}$, we see from \hyperref[eqn:specalg_b]{equation \ref*{eqn:specalg_b}}  that we perform 2 FLOPs (consisting of addition and division) for every $i \in [1, N-1]$. The initial element $b_1$ requires no FLOPs. Exactly the same holds for calculating $\vec{\tilde{g}}$ in \hyperref[eqn:specalg_g]{equation \ref*{eqn:specalg_g}}.

    However, for the backward substitution in \hyperref[eqn:specalg_v]{equation \ref*{eqn:specalg_v}}, we see that in addition to the $2(n-1)$ FLOPs, we need one more for the last term $v_n$.

    In total, we require

    \begin{equation}
      3 ~ \cdotp \big[ 2(n-1)\big] + 1 = 6n-5 \text{ FLOPs}
    \end{equation}

    \subsection*{Implement algorithm in code}

    Lastly, we write code that implements the algorithm. The code lies in \lstinline{Problem9.cpp}.

\end{document}
