\documentclass[english,notitlepage]{article}  % defines the basic parameters of the document
\usepackage[T1]{fontenc} %for å bruke æøå
\usepackage[utf8]{inputenc}
\usepackage{graphicx} %for å inkludere grafikk
\usepackage{mathpazo}
\usepackage[norsk]{babel}
% Standard stuff
\usepackage{amsmath,graphicx,varioref,verbatim,amsfonts,geometry,gensymb, multirow}
% colors in text
\usepackage[usenames,dvipsnames,svgnames,table]{xcolor}
% Hyper refs
\usepackage[colorlinks=true,allcolors=black]{hyperref}

\usepackage{caption}
\usepackage{enumitem}
\usepackage{tikz}             % draw figures manually
\usepackage{subfigure}        % imports a lot of cool and useful figure commands
\usepackage{float}
\usepackage{circuitikz}

\usepackage{csquotes}
\usepackage[backend=biber,style=alphabetic,sorting=ynt]{biblatex} %Imports biblatex package
\addbibresource{referanser.bib}

\begin{document}
\title{Project 1 - Problem 6}
\date{\today}

\maketitle

The general expression for $\boldsymbol{A}\vec{v} = \vec{g}$ with $\vec{a}$, $\vec{b}$ and $\vec{c}$ being the sub-, main- and supdiagonal respectively is:
\begin{equation}\label{eq:mat_Avg}
    \begin{bmatrix}
        b_1 & c_2 & 0 & 0 \\
        a_1 & b_2 & c_3 & 0 \\
        0 & a_2 & b_3 & c_4 \\
        0 & 0 & a_3 & b_4
    \end{bmatrix} \begin{bmatrix}
        v_1\\
        v_2\\
        v_3\\
        v_4
    \end{bmatrix} = \begin{bmatrix}
        g_1\\
        g_2\\
        g_3\\
        g_4
    \end{bmatrix}
\end{equation}

We can solve this with Gaussian elimination in two general steps: \hyperref[sec:forward]{[1]} Forward substitution and \hyperref[sec:backward]{[2]} Backward substitution.

\subsection*{Forward substitution}\label{sec:forward}

The goal here is to get an upper triangular matrix, that is we want to do the follwoing:
\begin{equation*}
    \begin{bmatrix}
        \cdot & \cdot & \cdot & \cdot \\
        \cdot & \cdot & \cdot & \cdot \\
        \cdot & \cdot & \cdot & \cdot \\
        \cdot & \cdot & \cdot & \cdot
    \end{bmatrix} \begin{bmatrix}
        \cdot\\
        \cdot\\
        \cdot\\
        \cdot
    \end{bmatrix} = \begin{bmatrix}
        \cdot\\
        \cdot\\
        \cdot\\
        \cdot
    \end{bmatrix}
    \rightarrow
    \begin{bmatrix}
        \cdot & \cdot & \cdot & \cdot \\
        0 & \cdot & \cdot & \cdot \\
        0 & 0 & \cdot & \cdot \\
        0 & 0 & 0 & \cdot
    \end{bmatrix} \begin{bmatrix}
        \cdot\\
        \cdot\\
        \cdot\\
        \cdot
    \end{bmatrix} = \begin{bmatrix}
        \cdot\\
        \cdot\\
        \cdot\\
        \cdot
    \end{bmatrix}
\end{equation*}

So we want to get rid of all $a_i$ entries in \hyperref[eq:mat_Avg]{the matrix equation (\ref*{eq:mat_Avg})}. We will number and denote the rows with roman numerals.

\begin{gather}
    \RN{2} = \RN{2} - \frac{a_2}{b_1}\RN{1}\\
    \text{The diagonal new entries will be denoted } \tilde{b_i} \nonumber \\
    \RN{3} = \RN{3} - \frac{a_3}{\tilde{b_2}}\RN{2}\\
    \RN{4} = \RN{4} - \frac{a_4}{\tilde{b_3}}\RN{3}\\
\end{gather}

This gives us the following algorithm for precdure
\begin{gather}\label{eq:for}
    \tilde{b_1} = b_1 \\
    \tilde{b_i} = b_i - \frac{a_i}{\tilde{b_{i-1}}}c_{i-1}\\\label{line:for_b}
    \tilde{g_1} = g_1 \\
    \tilde{g_i} = g_i - \frac{a_i}{\tilde{b_{i-1}}}\tilde{g_{i-1}}\\\label{line:for_g}
\end{gather}
This will be for $i = 2, 3, ..., n$ for a general tridiagonal matrix.

\subsection*{Backward substitution}\label{sec:backward}

Now that we have done the forward substitution we want to find the solution to all $v_i$ from the following equation:

\begin{equation}
    \begin{bmatrix}
        \tilde{b_1} & c_2 & 0 & 0 \\
        0 & \tilde{b_2} & c_3 & 0 \\
        0 & 0 & \tilde{b_3} & c_4 \\
        0 & 0 & 0 & \tilde{b_4}
    \end{bmatrix} \begin{bmatrix}
        v_1\\
        v_2\\
        v_3\\
        v_4
    \end{bmatrix} = \begin{bmatrix}
        \tilde{g_1}\\
        \tilde{g_2}\\
        \tilde{g_3}\\
        \tilde{g_4}
    \end{bmatrix}
\end{equation}

we quickly see that $v_4 = \frac{\tilde{g_4}}{\tilde{b_4}}$ and to get the remaining $v_i$ we move up with the following algorithm (we include the case of $v_n$ to generalize the algorithm):
\begin{gather}\label{eq:back}
    v_n = \frac{\tilde{g_n}}{\tilde{b_n}}\\
    v_i = \frac{\tilde{g_i} - c_iv_{i-1}}{\tilde{b_i}}\\
\end{gather}
for $i = n-1, n-2, ..., 2, 1$.

\subsection*{FLOPs}

We can see that \hyperref[eq:for]{algorithm \ref*{eq:for}} requires $2(n-1)$ floating point operations. $n-1$ FLOPs for \hyperref[line:for_b]{equation (\ref*{line:for_b})} and \hyperref[line:for_g]{equation (\ref*{line:for_g})} each. From \hyperref[eq:back]{equation \ref*{eq:back}} we can count $3(n-1) + 1$ FLOPs. This gives us a total of $5n - 4$ FLOPs to solve $\boldsymbol{A}\vec{v} = \vec{g}$ with a tridiagonal matrix of dimentions $n\times n$.

\end{document}