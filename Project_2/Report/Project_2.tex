\documentclass[english,notitlepage]{article}  % defines the basic parameters of the document
\usepackage[T1]{fontenc} %for å bruke æøå
\usepackage[utf8]{inputenc}
\usepackage{graphicx} %for å inkludere grafikk
\usepackage{mathpazo}
\usepackage[norsk]{babel}
% Standard stuff
\usepackage{amsmath,graphicx,varioref,verbatim,amsfonts,geometry,gensymb, multirow}
% colors in text
\usepackage[usenames,dvipsnames,svgnames,table]{xcolor}
% Hyper refs
\usepackage[colorlinks=true,allcolors=black]{hyperref}

\usepackage{caption}
\usepackage{enumitem}
\usepackage{tikz}             % draw figures manually
\usepackage{subfigure}        % imports a lot of cool and useful figure commands
\usepackage{float}
\usepackage{circuitikz}
\usepackage{listings}

\usepackage{csquotes}
\usepackage[backend=biber,style=alphabetic,sorting=ynt]{biblatex} %Imports biblatex package
\addbibresource{referanser.bib}

\title{FYS3150\\Project 2}
\author{Brage A. Trefjord\\Sigurd Sønvisen Vargdal\\Frida Oleivsgard Sørensen\\Nils Enric Canut Taugbøl}


\begin{document}

\maketitle
\textit{GitHub repository:} \texttt{\url{https://github.com/NilsECT/FYS3150/tree/main/Project_2/Code}}


\section*{Problem 1:}

We use the dimensionless variable $\hat{x} = \frac{x}{L}$, which gives us
$\frac{d \hat{x}}{dx} = \frac{1}{L}$. Using this, we get:

\begin{align*}
    \frac{d}{dx} = \frac{d\hat{x}}{dx} \frac{d}{d\hat{x}} = \frac{1}{L} \frac{d}{d\hat{x}}
\end{align*}

And we can find the scaled differential equation:

\begin{align*}
    \gamma \frac{d^2 u(x)}{dx^2} &= -F u(x)
    \\
    \gamma \left( \frac{1}{L} \frac{d}{d\hat{x}} \right)^2 u(\hat{x}) &= -F u(\hat{x})
    \\
    \gamma \frac{1}{L^2} \frac{d^2 u(\hat{x})}{d \hat{x}^2} &= -F u(\hat{x})
    \\
    \frac{d^2 u(\hat{x})}{d \hat{x}^2} &= - \frac{F L^2}{\gamma} u(\hat{x})
    \\
    \frac{d^2 u(\hat{x})}{d \hat{x}^2} &= - \lambda u(\hat{x}) \; , \hspace*{20pt} \lambda \equiv \frac{F L^2}{\gamma}
\end{align*}


\section*{Problem 2}

Our tridiagonal $6 \times 6$ matrix is

\begin{equation*}
    A = \frac{1}{h^2} \begin{bmatrix}
        2 & -1 & 0 & 0 & 0 & 0 \\
        -1 & 2 & -1 & 0 & 0 & 0 \\
        0 & -1 & 2 & -1 & 0 & 0 \\
        0 & 0 & -1 & 2 & -1 & 0 \\
        0 & 0 & 0 & -1 & 2 & -1 \\
        0 & 0 & 0 & 0 & -1 & 2
    \end{bmatrix}
\end{equation*}

where $h = \frac{1}{N+1}$, and $N = 6$ is the dimension of our square matrix.

We solve the equation

\begin{equation*}
    A \vec{v} = \lambda \vec{v}
\end{equation*}

both analytically and using armadillo in C++. Here $\lambda$ are the
eigenvalues, and $\vec{v}$ are the corresponding eigenvectors of $A$. See code
in repository. We find that the eigenvectors from armadillo are consistent with
those found analytically.


\section*{Problem 3}
\subsection*{a)}
See code in repository.

\subsection*{b)}


\section*{Problem 4}

See code in repository.


\section*{Problem 5}
\subsection*{a)}
(See code in repository.)\\
We have an $N \times N$ tridiagonal matrix. 

\begin{figure}[H]
    \centering
    \includegraphics[width=0.8\linewidth]{../Code/Problem_5_plot.pdf}
<<<<<<< HEAD
    \caption{Loglog plot of the number of transformations needed to compute the eigen values for an $N \times N$ matrix with a tolerance of 1e-10.}
=======
    \caption{Plot}
>>>>>>> a6c7f2a0f74d6a69a19f809898a31c7d51b49dd0
    \label{fig:5plot}
\end{figure}

\subsection*{b)}
For a Dense matrix we expect.. scaling behavior.

\section*{Problem 6}
\subsection*{a)}
See code in repository.

\begin{figure}[H]
    \centering
    \includegraphics[width=0.8\linewidth]{../Code/Problem_6_a_plot.pdf}
<<<<<<< HEAD
    \caption{Plot of the 3 lowest eigen vectors with the boundary conditions 0 at the start and the beginning for $n=10=>N=9$.}
    \label{fig:6aplot}
=======
    \caption{Plot}
    \label{fig:6aplot}
\end{figure}

\begin{figure}[H]
    \centering
    \includegraphics[width=0.8\linewidth]{../Code/Problem_6_b_plot.pdf}
    \caption{Plot}
    \label{fig:6bplot}
>>>>>>> a6c7f2a0f74d6a69a19f809898a31c7d51b49dd0
\end{figure}

\subsection*{b)}

\begin{figure}[H]
    \centering
    \includegraphics[width=0.8\linewidth]{../Code/Problem_6_b_plot.pdf}
    \caption{Plot of the 3 lowest eigen vectors with the boundary conditions 0 at the start and the beginning for $n=100=>N=99$.}
    \label{fig:6bplot}
\end{figure}

\end{document}