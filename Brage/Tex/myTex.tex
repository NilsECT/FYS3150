\documentclass[english,notitlepage]{revtex4-1}  % defines the basic parameters
%of the document For preview: skriv i terminal: latexmk -pdf -pvc filnavn



% if you want a single-column, remove reprint

% allows special characters (including æøå)
\usepackage[utf8]{inputenc}
%\usepackage[english]{babel}

%% note that you may need to download some of these packages manually, it
%depends on your setup. % I recommend downloading TeXMaker, because it includes
%a large library of the most common packages.

\usepackage{physics,amssymb}  % mathematical symbols (physics imports amsmath)
\usepackage{amsmath}
\usepackage{graphicx}         % include graphics such as plots
\usepackage{xcolor}           % set colors
\usepackage{hyperref}         % automagic cross-referencing (this is GODLIKE)
\usepackage{listings}         % display code
\usepackage{subfigure}        % imports a lot of cool and useful figure commands
\usepackage{float}
%\usepackage[section]{placeins}
\usepackage{algorithm}
\usepackage[noend]{algpseudocode}
\usepackage{subfigure}
\usepackage{tikz}
\usetikzlibrary{quantikz}
% defines the color of hyperref objects Blending two colors:  blue!80!black  =
% 80% blue and 20% black
\hypersetup{ % this is just my personal choice, feel free to change things
    colorlinks, linkcolor={red!50!black}, citecolor={blue!50!black},
    urlcolor={blue!80!black}}


\begin{document}

\title{Playing with Latex}      % self-explanatory
\author{Brage A. Trefjord}          % self-explanatory
\date{\today}                             % self-explanatory
\noaffiliation                            % ignore this, but keep it.

\maketitle

\section{Deriving the discretized Poisson equation.}

Let's start by deriving the discretized version of $\frac{d^2u}{dx^2}$. First,
we taylor expand the function $u(x+h)$ around the point $x$. We get:

\begin{align*}
    u(x+h) &= \sum_{n=0}^{\infty} \frac{u^{(n)}(x)}{n!} h^n
    \\
    u(x+h) &= u(x) + u'(x) h + \frac{1}{2} u''(x) h^2 + \frac{1}{6} u'''(x) h^3 + O_1(h^4)
\end{align*}

where $O_1(h^2)$ is the remainder of the expansion, and $O(h) \equiv
\frac{O_1(h^2)}{h}$. Now let's also Taylor expand $u(x-h)$ (again, around the
point $x$).

\begin{align*}
    u(x-h) &= \sum_{n=0}^{\infty} \frac{u^{(n)}(x)}{n!} (-h)^n
    \\
    u(x-h) &= u(x) - u'(x) h + \frac{1}{2} u''(x) h^2 - \frac{1}{6} u'''(x) h^3 + O_2(h^4)
\end{align*}

If we now add these two expansions together, we get:

\begin{align*}
    u(x+h) + u(x-h) &= u(x) + u'(x) h + \frac{1}{2} u''(x) h^2 + \frac{1}{6} u'''(x) h^3
    + O_1(h^4) 
    \\ & \hspace*{30pt} + u(x) - u'(x) h + \frac{1}{2} u''(x) h^2 - \frac{1}{6} u'''(x) h^3 + O_2(h^4)
    \\
    u(x+h) + u(x-h) &= 2u(x) + u''(x) h^2 + O_1(h^4) + O_2(h^4)
    \\
    u''(x) &= \frac{u(x+h) - 2u(x) + u(x-h)}{h^2} - \frac{O_1(h^4) + O_2(h^4)}{h^2}
    \\
    u''(x) &= \frac{u(x+h) - 2u(x) + u(x-h)}{h^2} + O(h^2)
\end{align*}

This can be discretized by letting $v_i \approx u(x)$, $v_{i+1} \approx u(x+h)$
and $v_{i-1} \approx u(x-h)$. Since this is an approximation we can ignore the
remainder, and our result becomes

\begin{equation*}
    v''_i = \frac{v_{i+1} - 2v_i + v_{i-1}}{h^2}
\end{equation*}

Here $x$ is discretized as $i h$, where $h$ is the step length, and $i$ is the
number of steps to reach the $x$ value. Using this, the discretization of the
forcing term $f(x) = 100 e^{-10 x}$ becomes $f_i = 100 e^{-10 ih}$. Using this
together with the discretized version of $v''(x)$ we get the complete
discretized Poisson equation:

\begin{equation}
    -\frac{v_{i+1} - 2v_i + v_{i-1}}{h^2} = 100 e^{-10 i h}
\end{equation}


\end{document}